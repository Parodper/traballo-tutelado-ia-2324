\chapter{Descripción das Características}
\label{chap:DescripciónDasCaracteristicas}

\section{TIBCO Cloud™ Integration Marketplace}

TIBCO Cloud™ Integration ten un espazo denominado «\textit{Marketplace}» onde ofrece acceso a conectores, complementos, extensións, aplicacións e aceleradores. Todas estas contribucións están desenvolvidas e proporcionadas por TIBCO, os socios de TIBCO, os provedores de software independentes e os usuarios de TIBCO Cloud™ Integration para axudar cos proxectos de integración de outros usuarios.

A \textit{Marketplace App Listing} é unha plataforma para compartir aplicacións de TIBCO Cloud Integration - Connect, TIBCO Flogo®, o TIBCO BusinessWorks™ con outros usuarios. Cando se crea un «\textit{Marketplace App Listing}» estase creando unha entrada no mercado de aplicacións de TIBCO Cloud Integration onde se inclúe información da aplicación.

Existen dous tipos:

\begin{description}
    \item[Privados] Só visibles para usuarios da túa organización.
    \item[Públicos] Visibles para todos os usuarios de TIBCO Cloud Integration.
\end{description}

\subsection{Crear un App Listing}

\begin{enumerate}
	\item Navega ata a páxina denominada Lista de Aplicacións (\textit{Apps List}) e abre a aplicación  \textit{IBCO Flogo} ou \textit{TIBCO Cloud™ Integration - Connect} que queres engadir ao Marketplace.
	
	\item Na páxina de detalles da aplicación (\textit{App Details}) seleccione a opción «Engadir listaxe do mercado» situado xunto ao nome da aplicación na parte superior da páxina.
	
	\item Completa os campos e selecciona "Publish privately" para mandala á Marketplace. (Por defecto todos os nuevos listings situasen como privados).
\end{enumerate}

Cando se crea unha aplicación o nome da organización é usado automaticamente na listaxe: se despois o nome da organización muda este campo non estará actualizado.

As aplicacións do \textit{Marketplace} pódense instalar na organización para usalos en integracións propias.

Pódese:

\begin{itemize}
    \item Instalar a App directamente dende o \textit{Marketplace}.
    \item Usar a utilizade de «\textit{Create App}» accesible dente a lista de aplicacións (\texit{Apps List})
\end{itemize}

Consideracións:

\begin{itemize}
    \item Cando se obtén unha app dende o \textit{Marketplace}, mostrase na listaxe de aplicacións co mesmo nome que a app usada para crear o listing, no o nome propio do listing.
    
    \item Obter a mesma aplicación varias veces engadirá un número ao nome da app cando  se amose nas Lista de Apps.
    
    \item Despois da instalación da app na organización, deberíanse abrir e actualizar manualmente as conexións de esquemas, as URL das API, etc.
\end{itemize}

\subsection{Obtención dunha App}

\begin{enumerate}
    \item Navegar ata o Marketplace e  selecciona o listing que desexase installar
    \item No panel dereito, seleccionar "GET" ou "REQUEST" dependendo de como o proveedor configurara o listing.
    
    \begin{description}
        \item[GET] instala a app inmediatamente.
        \item[REQUEST] mandase unha petición de acceso ao provedor. Se o provedor acepta a petición o botón cambiase a GET e xa se pode instalar a aplicación.
    \end{description}

	
	Despois de seleccionar GET para instalar a app, aparecese un pop-up dicindo que unha copia da app será instalada na organización, ademais de alertar sobre a información que será recollida e compartida co provedor da app.
	
	Para rematar, aceptar os termos e Crear a app (que agora mostrarase na Apps Lists), e na Apps List seleccionar unha nova app e completar a configuración.
\end{enumerate}

\section{Integración dirixida por API}

    A integración dirixida por API é un enfoque da integración que centrase no uso das APIs para conectar apps e datos. Coa integración dirixida por API, as empresas poden aproveitar as vantaxes das APIs para conectar apps de forma rápida e sinxela, independentemente de onde se sitúen esas aplicacións.
   
\subsection{Funcionamento}

\begin{enumerate}
    \item O desarrollador crea unha app de integración que utiliza APIs para conectarse ás apps que desexa integrar.
    \item A aplicación de integración utiliza as APIs para intercambiar datos entre as aplicacións.
    \item A aplicación de integración pode utilizar outros compoñentes, como fluxos de traballo, para automatizar tarefas e engadir funcionalidades adicionais.
\end{enumerate}

\subsection{Vantaxes}

\begin{description}
    \item[Velocidade:] Como xa se mencionou anteriormente, a integración dirixida pode axudar as empresas a conectar aplicacións de forma rápida e sinxela.
    \item[Flexibilidade:] Pode utilizarse para conectar apps de calquera tipo, independentemente de onde se encontren.
    \item[Reusabilidade:] As APIs poden reutilizarse en diferentes integracións, o que pode axudar ás empresas a aforrar tempo e costes. 
\end{description}

\subsection{Para que se pode utilizar}

\begin{description}
    \item[Integración de aplicacions na nube:] A integración dirixida por APIs pode utilizarse para conectar apps que se encontran na nube. Exemplo: AWS, Salesforce...
    
    \item[Integración de datos:] A integración dirixida por APIs pode utilizarse para integrar datos pertencentes a diferentes fontes. Exemplos: Bases de datos, ficheiros...
\end{description}

\section{TIBCO Cloud™ Integration - API Modeler}

API Modeler é unha ferramenta web sinxela de usar que permite crear e modelar APIs REST de forma visual.

\subsection{Características principais:}

\begin{itemize}
    \item[Importar e editar especificacións de APIs:] Importa sen problemas especificacións de API REST existentes en formato YAML o JSON para o seu posterior refinamento e edición.
    
    \item[Modelo Visual de API:] Utiliza a intuitiva interface visual para o modelaxe da API REST, incluídas as definicións de recursos, operacións e tipos de datos.
    
    \item[Mocking e Implementacion de APIs:] Aproveita a función de mocking de API para simular o comportamento de API antes de despregala, asegurando a súa funcionalidade.
\end{itemize}

\section{TIBCO Cloud™ Integration - API Mock App}

Unha aplicación API Mock é unha maqueta dunha aplicación que se crea a partir dunha especificación de API existente. Utilizase para simular o comportamento de unha API. 

\subsection{Obxectivo}

\begin{description}
    \item[Probas de Integración:] Pódese utilizar para probar integracións que utilizan APIs. Esto pode axudar a garantir que as integracións funcionen correctamente.
    \item[Desenvolvemento de Integracións:] Pódese utilizar para o desenvolvemento de integracións que utilizan APIs para axudar a crear integracións máis rapidamente e reducir o risco de erros.
\end{description}

\subsection{Compoñentes}

\begin{description}
    \item[Endpoints:] Enderezo URL que se utiliza para acceder á API Mock App
    \item[Definición de API:] A definición de API especifica os métodos da API, os parametros que aceptan e os datos que devolven.
    \item[Implementación da API:] Código que implementa os metodos da API
\end{description}

\section{Integración dirixida por eventos}

A Integración dirixida por eventos é un enfoque que se centra no uso de eventos para conectar aplicacións e datos. As empresas poden aproveitar as vantaxes dos eventos para conectar de forma rápida e sinxela, independentemente de onde se encontren estas aplicacións.

TIBCO Cloud™ Integration provee unha gran variedade de funcionalidades que facilitan a integración dirixida por eventos:

\begin{itemize}
    \item Amplia Gama de eventos
    \item Ferramenta de desenvolvemento de eventos: Estas ferramentas axudan aos de desenvolvemento a crear eventos seguros, escalables e doados de usar.
    \item Xentión de eventos: Facilitan o monitoreo, mantemento e a administración de eventos garantizando que sexan seguros e eficaces.
\end{itemize}

\subsection{Funcionamento}

\begin{enumerate}
    \item O desenvolvedor crea o evento que se utiliza para conectar as Apps que desexa integrar. Un evento é unha mensaxe que conten información sobre un cambio que se produciu nunha aplicación ou sistema. O evento pode conter información sobre o tipo de cambio, a data e hora ou os datos afectados.
    
    Para a creación deste evento pódese utilizar a ferramenta de desenvolvemento de eventos de TIBCO Cloud™ Integration que ten interfaz gráfica.
    
    \item O evento publicase nun bus de eventos, é dicir, nun servicio que permite as aplicacións intercambiar eventos.\\
    Concretamente o bus de eventos de TIBCO Cloud™ Integration denominase Event Broker.
    \item As aplicacións que están suscritas ao bus de eventos reciben o evento. Isto realizase na ferramenta de de desenvolvemento de fluxos de integración.
    \item As aplicacións procesan o evento realizando acción en función dos seus contidos. Isto tamén realizase coa ferramenta de desenvolvemento de fluxos de integración.
\end{enumerate}

\section{Fluxos de Integración nativos na nube}

TIBCO Cloud™ Integration ofrece unha amplia gama de funcionalidades para crear e xestionar fluxos de integración nativos na nube.

\subsection{Creación de fluxos de Integración}

A creación de fluxos de integracion é un proceso sinxelo que se pode realizar mediante uncha interfaz de usuario intuitiva. Os fluxos de integraicón podense crear a partir dunha plantilla ou desde cero.\\
As plantillas proporcionan un punto de partida para crar fluxos de integracións comunes. Por exemplo existen plantillas disponibles para replicar datos e integrar aplicacións ou procesar eventos.

\subsection{Transformacións de datos}

TIBCO Cloud™ Integration ten uha variedade de funcións de transformacións de datos, como a limpeza, a transformación e a validación. Permiten garantizar a calidade de datos que se moven a traves dos fluxos de integración.

\subsection{Enrutamento de datos}

TIBCO Cloud™ Integration permite enrutar os datos a través dunha variedade de destinos, como aplicaciones, bases de datos, servicios na nube ou sistemas de almacenamento.
O enrutamento podese configurar utilizando regras o funcións. As regras utilizanse para enrutar os datos en función de criterios específicos e as funcións utilizanse para enrutar os datos en función da lóxica definida polo usuario.

\subsection{Monitorización ou Xestión}

TIBCO Cloud™ Integration ten unha gran variedade de capacidades de monitorización e Xestion o que permite aos usuarios supervisar o rendemento de fluxos de integración e solucionar problemas.

\subsection{TIBCO Cloud™ Integration - Connect}

TIBCO Cloud™ Integration - Connect é  unha solución de integración de datos que permite a conectividade sen problemas entre aplicacións basadas na nube e locais.\\
Os integradores teñen a capacidade de crear e administrar integracións de datos robustas cunha interfaz de usuario doada de entender e un conxunto comppleto de conectores.\\

\subsubsection{Compoñentes Principais}

\begin{itemize}
    \item \textbf{Motor de Integración:} Corazón de TIBCO Cloud™ Integration - Connect responsable de dirixir o movemento de datos entre aplicacións de fontes de datos.\\
    Encargase das transformacións complexas, manexo de erros e enrutamento de datos.
    \item \textbf{Conectores:} Provee contectividade lista para usar nunha gran gama de aplicacións, bases de datos e plataformas na nube. Cada conector encapsula os protocolos de comunicación e formatos de datos específicos do sistema conectado.
    \item \textbf{Axente: }Compoñente do software local instalado que facilita a comunicación segura entre TIBCO Cloud™ Integration - Connect na nube e as fontes de datos locais. Garantiza a seguridade dos datos e o cumprimento das políticas de firewall.
\end{itemize}

\subsubsection{Ventaxes }

\begin{itemize}
    \item Conectividade Amplia: admite unha amplia gama de aplicacións, bases de datos e plataformas na nube, o que permite unha integración sen problemas en entornos de TI diversos.
    \item Mellora da calidade dos datos: Facilita a limpeza de datos, transformación e validación para garantizar a precisión e consistencia dos datos nos sistemas.
    \item Reducción dos costos de integración: Elimina a necesidade da manipulación manual dos datos e reduce o tempo e os recursos necesarios para o mantemento da integración.
    \item Escalabilidade: Admite volúmenes de datos crecentes e complexidaded de integración, asegurando que a integración de datos pode escalar cunhos requisitos comerciais en evolución.
\end{itemize}
